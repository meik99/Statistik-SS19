\documentclass{article}

\usepackage{Sweave}
\begin{document}
\Sconcordance{concordance:UEbung2.tex:UEbung2.Rnw:%
1 2 1 1 0 8 1 1 4 3 0 3 1 7 0 1 1 8 0 1 2 2 1 1 2 1 0 1 3 2 0 3 1 4 0 2 %
2 1 0 1 1 6 0 1 1 5 0 5 1 4 0 1 2 3 1 1 4 3 0 1 3 2 0 1 2 4 1 1 9 6 0 1 %
2 2 1 5 0 1 3 2 1 1 2 1 0 1 3 2 0 1 2 2 1 1 6 4 0 2 2 2 1 4 0 1 2 3 1 1 %
3 2 0 2 1 3 0 2 2 8 0 1 1 8 0 1 2 2 1 1 2 1 0 1 1 4 0 2 2 1 0 1 1 6 0 1 %
2 4 1 4 0 1 2 4 1 1 2 1 0 2 1 1 2 2 1 1 2 2 1 2 2 2 1 9 0 1 2 1 3 1 2 4 %
1 1 2 1 0 2 1 1 2 2 1 1 2 2 1 2 2 2 1 9 0 1 2 1 3 1 2 3 1 1 4 3 0 1 4 2 %
0 1 2 3 1 1 8 6 0 1 1 12 0 1 2 4 1 1 2 7 0 1 2 2 1 1 2 7 0 1 2 2 1 1 2 %
7 0 1 2 2 1}




\section{Übung 2}
\subsection{Aufgabe 9}
\subsubsection{a}

\begin{Schunk}
\begin{Sinput}
> income <- c(2, 4, 6, 4, 7, 5, 7, 4, 3, 5,
+             5, 8, 6, 3, 5, 2, 9, 4, 5, 6, 
+             8, 3, 10, 5, 4, 3, 7, 4, 6, 4)
> income.table <- table(income)
> income.proptable <- prop.table(table(income))
> addmargins(income.table)
\end{Sinput}
\begin{Soutput}
income
  2   3   4   5   6   7   8   9  10 Sum 
  2   4   7   6   4   3   2   1   1  30 
\end{Soutput}
\begin{Sinput}
> round(income.proptable * 100, 2)
\end{Sinput}
\begin{Soutput}
income
    2     3     4     5     6     7     8     9    10 
 6.67 13.33 23.33 20.00 13.33 10.00  6.67  3.33  3.33 
\end{Soutput}
\end{Schunk}

\subsubsection{b}

\begin{Schunk}
\begin{Sinput}
> library("graphics")
> income <- c(2, 4, 6, 4, 7, 5, 7, 4, 3, 5,
+             5, 8, 6, 3, 5, 2, 9, 4, 5, 6, 
+             8, 3, 10, 5, 4, 3, 7, 4, 6, 4)
> income.table <- table(income)
> income.proptable <- prop.table(table(income))
> barplot(income.table)
\end{Sinput}
\end{Schunk}
\includegraphics{UEbung2-002}

\begin{Schunk}
\begin{Sinput}
> income.cumulatedSum <- cumsum(income.table)
> income.cumulatedSum
\end{Sinput}
\begin{Soutput}
 2  3  4  5  6  7  8  9 10 
 2  6 13 19 23 26 28 29 30 
\end{Soutput}
\begin{Sinput}
> names(income.cumulatedSum)
\end{Sinput}
\begin{Soutput}
[1] "2"  "3"  "4"  "5"  "6"  "7"  "8"  "9"  "10"
\end{Soutput}
\begin{Sinput}
> plot(income.cumulatedSum, type = "l", axes=FALSE, col="blue", ylab = "Kumulierte Häufigkeit", xlab = "Umsatz")
> axis(1, at=1:length(names(income.cumulatedSum)), labels=names(income.cumulatedSum))
> axis(2, at=1:tail(income.cumulatedSum, n=1), labels=1:tail(income.cumulatedSum, n=1))
> abline(v=1:length(names(income.cumulatedSum)), col="lightgray")
> abline(h=1:max(income.cumulatedSum), col="lightgray")
\end{Sinput}
\end{Schunk}
\includegraphics{UEbung2-003}

\subsubsection{c [1,5 ; 3,5), [3,5 ; 5,5), [5,5 ; 7,5),[7,5 ; 9,5), [9,5 ; 11,5)}


\begin{Schunk}
\begin{Sinput}
> #[1,5 ; 3,5), [3,5 ; 5,5), [5,5 ; 7,5),[7,5 ; 9,5), [9,5 ; 11,5) und[1,5 ; 3,5), [3,5 ;6,5),[6,5 ; 10,5) 
> 
> library("graphics")
> income <- c(2, 4, 6, 4, 7, 5, 7, 4, 3, 5,
+             5, 8, 6, 3, 5, 2, 9, 4, 5, 6, 
+             8, 3, 10, 5, 4, 3, 7, 4, 6, 4)
> income.class1 <- income[income >= 1.5 & income < 3.5]
> income.class2 <- income[income >= 3.5 & income < 5.5]
> income.class3 <- income[income >= 5.5 & income < 7.5]
> income.class4 <- income[income >= 7.5 & income < 9.5]
> income.class5 <- income[income >= 9.5 & income < 11.5]
> income.frequency <- c(
+   length(income.class1),
+   length(income.class2),
+   length(income.class3),
+   length(income.class4),
+   length(income.class5)
+ )
> hist(income, breaks = c(1.5, 3.5, 5.5, 7.5, 9.5, 11.5), axes = FALSE, freq = TRUE)
> axis(1, at = c(1.5, 3.5, 5.5, 7.5, 9.5, 11.5), labels = c(1.5, 3.5, 5.5, 7.5, 9.5, 11.5))
> axis(2, at = c(0, income.frequency), labels = c(0, income.frequency))
> 
\end{Sinput}
\end{Schunk}
\includegraphics{UEbung2-004}

\subsubsection{c [1,5 ; 3,5), [3,5 ;6,5),[6,5 ; 10,5)}

\begin{Schunk}
\begin{Sinput}
> library("graphics")
> income <- c(2, 4, 6, 4, 7, 5, 7, 4, 3, 5,
+             5, 8, 6, 3, 5, 2, 9, 4, 5, 6, 
+             8, 3, 10, 5, 4, 3, 7, 4, 6, 4)
> income.class6 <- income[income >= 1.5 & income < 3.5]
> income.class7 <- income[income >= 3.5 & income < 6.5]
> income.class8 <- income[income >= 6.5 & income < 10.5]
> income.frequency <- c(
+   length(income.class6),
+   length(income.class7),
+   length(income.class8)
+ )
> breaks <- c(1.5, 3.5, 6.5, 10.5)
> hist(income, breaks = breaks, axes=FALSE, freq = TRUE)
> axis(1, at=breaks, labels=breaks)
> axis(2, at=c(0, income.frequency), labels = c(0, income.frequency))
\end{Sinput}
\end{Schunk}
\includegraphics{UEbung2-005}

\subsection{Aufgabe 10}
\subsubsection{a}

\begin{Schunk}
\begin{Sinput}
> children <- c(0, 2, 0, 2, 0, 0, 1, 2, 0, 0, 2, 1, 2, 1, 2, 1, 1, 1, 1, 6, 
+               1 ,2, 0, 2, 0, 0, 1, 0, 0, 1, 0, 0, 1, 0, 0, 1, 1, 0, 2, 1)
> children.table <- table(children);
> children.proptable <- prop.table(children.table)
\end{Sinput}
\end{Schunk}

\begin{Schunk}
\begin{Sinput}
> addmargins(children.table)
\end{Sinput}
\begin{Soutput}
children
  0   1   2   6 Sum 
 16  14   9   1  40 
\end{Soutput}
\begin{Sinput}
> addmargins(children.proptable * 100)
\end{Sinput}
\begin{Soutput}
children
    0     1     2     6   Sum 
 40.0  35.0  22.5   2.5 100.0 
\end{Soutput}
\end{Schunk}

\subsubsection{b}

\begin{Schunk}
\begin{Sinput}
> barplot(children.table, axes = FALSE)
> axis(2, at=0:max(children.table), labels = 0:max(children.table))
\end{Sinput}
\end{Schunk}
\includegraphics{UEbung2-008}

\begin{Schunk}
\begin{Sinput}
> children.cumsum <- cumsum(children.proptable) * 100
> children.cumsum
\end{Sinput}
\begin{Soutput}
    0     1     2     6 
 40.0  75.0  97.5 100.0 
\end{Soutput}
\begin{Sinput}
> plot(x = children.cumsum, type = "l", col="blue", axes = FALSE, xlab = "", ylab = "")
> axis(1, at = 1:length(children.cumsum), labels = names(children.cumsum))
> axis(2, at = children.cumsum, labels = children.cumsum)
> abline(h = children.cumsum, col="lightgray")
> abline(v = 1:length(children.cumsum), col="lightgray")
\end{Sinput}
\end{Schunk}
\includegraphics{UEbung2-009}

\subsection{Aufgabe 11}

Therapie A

\begin{Schunk}
\begin{Sinput}
> A.small.all <- 87
> A.small <- 81
> A.small.non <- A.small.all - A.small
> A.big.all <- 263
> A.big <- 192
> A.big.non <- A.big.all - A.big
> A.all <- 350
> A <- 273
> A.non <- A.all - A
> A.prop <- c(A.small / A.all, A.small.non / A.all, A.big / A.all, A.big.non / A.all)
> names(A.prop) <- c("Klein Nierensteine - Erfolg", "Kleine Nierensteine - Kein Erfolg", "Große Nierensteine - Erfolg", "Große Nierensteine - Kein Erfolg")
> A.prop <- round(A.prop * 100, 2)
> A.prop
\end{Sinput}
\begin{Soutput}
      Klein Nierensteine - Erfolg Kleine Nierensteine - Kein Erfolg 
                            23.14                              1.71 
      Große Nierensteine - Erfolg  Große Nierensteine - Kein Erfolg 
                            54.86                             20.29 
\end{Soutput}
\end{Schunk}

\includegraphics{UEbung2-011}



Therapie B

\begin{Schunk}
\begin{Sinput}
> B.small.all <- 270
> B.small <- 234
> B.small.non <- B.small.all - B.small
> B.big.all <- 80
> B.big <- 55
> B.big.non <- B.big.all - B.big
> B.all <- 350
> B <- 289
> B.non <- B.all - B
> B.prop <- c(B.small / B.all, B.small.non / B.all, B.big / B.all, B.big.non / B.all)
> names(B.prop) <- c("Klein Nierensteine - Erfolg", "Kleine Nierensteine - Kein Erfolg", "Große Nierensteine - Erfolg", "Große Nierensteine - Kein Erfolg")
> B.prop <- round(B.prop * 100, 2)
> B.prop
\end{Sinput}
\begin{Soutput}
      Klein Nierensteine - Erfolg Kleine Nierensteine - Kein Erfolg 
                            66.86                             10.29 
      Große Nierensteine - Erfolg  Große Nierensteine - Kein Erfolg 
                            15.71                              7.14 
\end{Soutput}
\end{Schunk}

\includegraphics{UEbung2-013}

\subsection{Aufgabe 12}
\subsubsection{a}

\begin{Schunk}
\begin{Sinput}
> classes.calcWidth <- function(begin, end){
+   abs(end - begin)
+ }
> classes.calcMiddle <- function(begin, end){
+   (begin + end) / 2
+ }
> taxpayers <- c(47996, 191492, 124498, 104428, 67988, 31125)
> taxpayers.sum <- sum(taxpayers)
> classes.begin <- c(0, 10, 20, 30, 50)
> classes.end <- c(10, 20, 30, 50, 100)
> classes <- data.frame(
+   Klassen = c("< 10", ">= 10 < 20", ">= 20 < 30", ">= 30 < 50", ">= 50 < 100", " >= 100"),
+   Steuerpflichtige = taxpayers,
+   Breite = c(classes.calcWidth(classes.begin, classes.end), "unbk."),
+   Mitte = c(classes.calcMiddle(classes.begin, classes.end), "unbk."),
+   Häufigkeit = round(taxpayers / taxpayers.sum * 100, 2)
+ )
> classes
\end{Sinput}
\begin{Soutput}
      Klassen Steuerpflichtige Breite Mitte Häufigkeit
1        < 10            47996     10     5       8.46
2  >= 10 < 20           191492     10    15      33.74
3  >= 20 < 30           124498     10    25      21.94
4  >= 30 < 50           104428     20    40      18.40
5 >= 50 < 100            67988     50    75      11.98
6      >= 100            31125  unbk. unbk.       5.48
\end{Soutput}
\end{Schunk}

\subsubsection{b}

Prozentsatz der Fälle mit einer Merkmalsausprägung kleiner 30.000 Euro

\begin{Schunk}
\begin{Sinput}
> cat(sum(classes[c(1,2,3),]$Häufigkeit), "%")
\end{Sinput}
\begin{Soutput}
64.14 %
\end{Soutput}
\end{Schunk}

Prozentsatz der Fälle mit einer Merkmalsausprägung von mindestens 10.000 Euro bis höchstens 50.000 Euro 

\begin{Schunk}
\begin{Sinput}
> cat(sum(classes[c(2,3,4),]$Häufigkeit), "%")
\end{Sinput}
\begin{Soutput}
74.08 %
\end{Soutput}
\end{Schunk}

Prozentsatz der Fälle mit einer Merkmalsausprägung größer 50.000 Euro

\begin{Schunk}
\begin{Sinput}
> cat(sum(classes[c(5,6),]$Häufigkeit), "%")
\end{Sinput}
\begin{Soutput}
17.46 %
\end{Soutput}
\end{Schunk}


\end{document}
